\documentclass[12pt, a4paper, simple]{eskdtext}

\usepackage{hyperref}
\usepackage{env}
\usepackage{_sty/gpi_lst}
\usepackage{_sty/gpi_toc}
\usepackage{_sty/gpi_t}
\usepackage{_sty/gpi_p}
\usepackage{_sty/gpi_u}

% Код
\ESKDletter{}{К}{Р}
\def \gpiDocTypeNum {81}
\def \gpiDocVer {00}
\def \gpiCode {\ESKDtheLetterI\ESKDtheLetterII\ESKDtheLetterIII.\gpiStudentGroupName\gpiStudentGroupNum.\gpiStudentCard-0\gpiDocNum~\gpiDocTypeNum~\gpiDocVer}

\def \gpiDocTopic {ПОЯСНИТЕЛЬНАЯ ЗАПИСКА}

% Графа 1 (наименование изделия/документа)
\ESKDcolumnI {\ESKDfontIII \gpiTopic \\ \gpiDocTopic}

% Графа 2 (обозначение документа)
\ESKDsignature {\gpiCode}

% Графа 9 (наименование или различительный индекс предприятия) задает команда
\ESKDcolumnIX {\gpiDepartment}

% Графа 11 (фамилии лиц, подписывающих документ) задают команды
\ESKDcolumnXIfI {\gpiStudentSurname}
\ESKDcolumnXIfII {\gpiTeacherSurname}
\ESKDcolumnXIfV {\gpiTeacherSurname}

\begin{document}
    \begin{ESKDtitlePage}
    \begin{center}
        \gpiMinEdu \\
        \gpiEdu \\
        \gpiKaf \\
    \end{center}

    \vfill

    \begin{flushright}
        \begin{minipage}[t]{.45\textwidth}
            <<К защите допускаю>> \\
            \gpiHeadDepartmentInfo \\
            \underline{\hspace{3cm}} \gpiHeadDepartmentName~\gpiHeadDepartmentSurname \\
            \PageTitleDateField
        \end{minipage}
    \end{flushright}

    \vfill

    \begin{center}
        \gpiTopic \\
    \end{center}

    \vfill

    \begin{center}
        \textbf{\gpiDocTopic} \\
        ПО ДИСЦИПЛИНЕ \gpiDiscipline \\
    \end{center}

    \vfill

    \begin{center}
        \gpiCode \\
        Листов \pageref{LastPage} \\
    \end{center}

    \vfill

    \begin{flushright}
        \begin{minipage}[t]{.49\textwidth}
            \begin{minipage}[t]{.75\textwidth}
                \begin{flushright}
                    Руководитель

                    \hspace{0pt}

                    Выполнил

                    \hspace{0pt}

                    Консультанты:

                    по ЕСПД

                    Рецензент
                \end{flushright}
            \end{minipage}
        \end{minipage}
        \begin{minipage}[t]{.49\textwidth}
            \begin{flushright}
                \begin{minipage}[t]{.75\textwidth}
                    \gpiTeacherName~\gpiTeacherSurname

                    \hspace{0pt}

                    \gpiStudentName~\gpiStudentSurname

                    \hspace{0pt}

                    \hspace{0pt}

                    \gpiTeacherName~\gpiTeacherSurname

                    \gpiReviewerName~\gpiReviewerSurname
                \end{minipage}
            \end{flushright}
            
        \end{minipage}
    \end{flushright}

    \vfill

    \begin{center}
        \ESKDtheYear
    \end{center}
\end{ESKDtitlePage}


    \ESKDthisStyle{empty}
    лист с заданием
    \newpage
    \ESKDthisStyle{formII}

    % Содержание
    \tableofcontents                                
    \paragraph{ПРИЛОЖЕНИЕ А. СХЕМА ПРОГРАММЫ}
    \paragraph{ПРИЛОЖЕНИЕ Б. ТЕКСТ ПРОГРАММЫ}
    \newpage

    %
    \newpage
    \addcontentsline{toc}{section}{ВВЕДЕНИЕ}
    \section*{ВВЕДЕНИЕ}
    \newpage

    %
    \section{АНАЛИЗ ПОСТАВКИ ЗАДАЧИ}
    \subsection{Перечень функций}
    \subsection{Требования пользователей}
    \subsection{Описание предметной области}
    \subsection{Варианты использования программы в виде диаграмм прецедентов (use case)}
    \subsection{Первичное описание объектов и классов, прецедентов системы}
    \subsection{Первоначаьное описание отношений между классами}
    \subsection{Диаграмма состояний (statechart diagram) для прецендентов}
    \newpage

    %
    \section{ПРОЕКТИРОВАНИЕ СТРУКТУРЫ ПРИЛОЖЕНИЯ}
    \subsection{Диаграммы классов предметной области}
    \subsection{Графический интерфейс приложения}
    \subsection{Общая диаграмма с учётом каркаса}
    \subsection{Диаграмма последовательностей (sequence diagram)}
    \subsection{Диаграмма видов деятельности (activity diagram)}
    \newpage
    
    %
    \section{РАЗРАБОТКА АЛГОРИТМОВ ФУНКЦИОНИРОВАНИЯ И СТРУКТУР ДАННЫХ}
    \newpage
    
    %
    \section{РЕАЛИЗАЦИЯ ПРИЛОЖЕНИЯ И РЕЗУЛЬТАТЫ ИСПЫТАНИЙ}
    \subsection{Диаграмма компонентов (component diagram)}
    \subsection{Диаграмма развёртывания (deployment diagram)}
    \subsection{Тестирование приложения}
    \newpage

    %
    \newpage
    \addcontentsline{toc}{section}{СПИСОК ИСПОЛЬЗОВАННЫХ ИСТОЧНИКОВ}
    \section*{СПИСОК ИСПОЛЬЗОВАННЫХ ИСТОЧНИКОВ}
    \begin{enumerate}
        \item[1.] title- [Электронный ресурс]
        Режим доступа: \url{https://www.youtube.com/watch?v=}
        Дата~доступа:~xx.xx.2022.
    \end{enumerate}
    \newpage
\end{document}
